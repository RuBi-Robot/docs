%!TEX root = ../../../report.tex
\section{Machining} % (fold)
\label{sec:machining}
This section presents the process of machining followed for the non 3D-printed components that required some extra treatment during the assembly of the robot.
The need of use of these parts is justified by the fact that the 3D printers used have a limited printing volume.
Additionally, the mechanical properties given by the PLA did not satisfy all the requirements of all the parts.
The rods of the joint axes, the beams used to create the holding structure or the carbon fiber tubes are some examples of these requirements of size and mechanical constraints.

The original carbon fiber tubes have been manually cut and drilled to give them their final configuration.
Special attention has been paid to the symmetry of the frame during the construction following the design criteria established in \ref{sec:physical_properties}.
The joints axes rods have been also manually cut and filed according to the design.
Besides, the holding structure mounted over the treadmill has been cut and assembled by hand as well.
Knowing that this kind of operations is human error prone, the number of parts on the frame that needed to be manually produced has been tried to be minimized during the design.
The drawings for such parts are included as appendices in \ref{app:mechanical_drawings}.

% section machining (end)