%!TEX root = ../../report.tex
\newenvironment{resumen}%
    {\null\vfill\begin{center}%
    \bfseries Resumen\end{center}}%
    {\vfill\null}
        \begin{resumen}
        El moderno campo de la robótica, en su intento por comprender e imitar los mecanismos que dan lugar a la locomoción bípeda, ha alumbrado en las últimas décadas a algunas de las criaturas bípedas más avanzadas, inspirándose en la naturaleza y el ser humano.
        Este proyecto de master quiere aportar a esta línea de investigación en ingeniería el robot bípedo RuBi y su entorno de desarrollo.
        RuBi es un robot de proporciones humanas, bajo coste y amortiguación regulable en las articulaciones concebido para la investigación en el control y generación de distintos tipos de locomoción bípedo.
        Conforman el conjunto de intrumentos creado las piernas del robot hasta la cadera, su sistema de control basado en ROS, su modelo de simulación en Gazebo y un banco de pruebas para desplazamientos en 2D.
        La interfaz con los controladores del robot y su entorno de simulación ha sido implementada y testeada con éxito.
        Además, se han diseñado varios experimentos para el prototipo del robot, aunque no han podido llevarse a cabo debido a los e imprevistos retrasos sufridos.
        Por último, todo ha sido debidamente documentado para facilitar a los futuros usuarios el uso del proyecto.
        \end{resumen}