%!TEX root = ../../../report.tex
\section{Robot model definition} % (fold)
\label{sec:robot_definition}
As in May of 2016, ROS Jade ecosystem offers three different formats to describe a robot: SDF, URDF and Xacro.
During the process of actual construction of the defined robot model they are all transformed to a unique low-level format, but the selection of the syntax to describe the robot for the simulation is needed from the beginning.

\begin{enumerate}
  \item \textbf{URDF\footnote{http://wiki.ros.org/urdf}}: is an open standard used in all the simulators mentioned or others like RobWork \cite{robwork}. 
  It allows to define all the properties of a single robot but lacks other which are important when simulating. 
  It is mainly used for visual representations or schematic robot definitions.
  \item \textbf{Xacro\footnote{http://wiki.ros.org/xacro}}: is a parametric format that facilitates the writing of the URDF.
  It allows logic conditions like \textit{if} or \textit{for} and from ROS Jade  virtually any python condition can be used.
  This code is then compiled into a URDF automatically which gives Xacro complete interoperability with URDF readers.
  \item \textbf{SDF\footnote{http://sdformat.org}}: adapted to the current requirements of the simulation environments.
  With it can be defined from \textit{worlds} to air properties in the case of UAV simulations for instance.
\end{enumerate}

Despite SDF contains more information, there are several tools in ROS Jade that make use of URDF.
Two of them are RViz and ROS Control, explained in the section \ref{sub:ros_control}.
As this last one is a pillar of the framework of Rubi, the decision of making use of URDF as format for describing the robot was taken.
However, the robot developed have been written in Xacro due to the wide range of existing development tools.
% section robot_definition (end)